\documentclass[10pt]{article}
\usepackage[top=1.5cm, bottom=.5cm, left=1.5cm, right=1.5cm]{geometry}
\usepackage{array, xcolor}
\usepackage{bibentry}
\usepackage{natbib}
\usepackage[T1]{fontenc}
\usepackage[brazil]{babel}
\usepackage{colortbl}
\usepackage{hyperref}
\usepackage[utf8]{inputenc}
\definecolor{lightgray}{gray}{0.8}
\newcolumntype{L}{>{\raggedleft}p{0.14\textwidth}}
\newcolumntype{R}{p{0.8\textwidth}}
\newcommand\VRule{\color{lightgray}\vrule width 0.5pt}

\begin{document}
\pagestyle{empty}
\begin{minipage}[ht]{0.48\textwidth}
\begin{flushleft}
\small{Scripps Institution of Oceanography} \\
\small{University of California, San Diego}\\
\small{9500 Gilman Drive \#0208} \\
\small{La Jolla, CA 92093} \\
\end{flushleft}
\end{minipage}
\hfill
\begin{minipage}[ht]{0.48\textwidth}
\begin{flushright}
\noindent \href{mailto:avillasboas@ucsd.edu}{avillasboas@ucsd.edu} \ \ \ \ \ \ \ \ \ \ \ \ \ \ \ \ \    \\
{\url{https://github.com/biavillas} }\\
\small{+1 858 848--9283} \ \ \ \ \ \ \ \ \ \ \ \ \ \ \ \ \  \ \ \ \ \ \ \ \ \ \   \\
\end{flushright}
\end{minipage}


\vspace{.5cm}
\begin{center}
	{\bfseries\Huge Bia Villas B\^{o}as}
\end{center}
\vspace{.5cm}

\section*{Education}
\vspace{.3cm}
\begin{tabular}{L!{\VRule}R}
\textsc{2014--Present} & \textbf{PhD in physical oceanography}, Scripps Institution of Oceanography. \\ 
\end{tabular}
\newline \noindent
\newline \noindent
\begin{tabular}{L!{\VRule}R}
\textsc{2012--2014} & \textbf{MSc. in physical oceanography}, University of São Paulo. \\ 
\end{tabular}
\newline \noindent
\newline \noindent
\begin{tabular}{L!{\VRule}R}
\textsc{2007--2011} & \textbf{BSc. in physics}, Federal University of Rio Grande do Norte. \\
\end{tabular}

\section*{Research Experience}
\vspace{.3cm}
\begin{tabular}{L!{\VRule}R}
\textsc{2014--Present} &{\bf Graduate Student Researcher - SIO }\\
& I look at how surface currents modulate the wave field at meso and submesoscales, and how non-breaking waves contribute to vertical mixing in the upper ocean. Advisors: \textbf{Sarah Gille}, Matthew Mazloff, and Bruce Cornuelle.\\[5pt]
\end{tabular}
\newline \noindent 
\newline \noindent
\begin{tabular}{L!{\VRule}R}
\textsc{2012--2014} &{\bf Graduate Student Researcher - IOUSP }\\
& Masters student at the Oceanographic Institute of the University of São Paulo (IOUSP) working on air--sea interactions at mesoscales. Title of the project: \textit{\textbf{``The contribution of mesoscale eddies to the surface heat budget in the South Atlantic''}}, funded by the São Paulo Research Foundation (FAPESP)\\[5pt]
\end{tabular}
\newline \noindent 
\newline \noindent
\newline \noindent
\begin{tabular}{L!{\VRule}R}
\textsc{2013} &{\bf Visiting Research Student - LEGOS}\\
& Visiting research student at the \textit{Laboratoire d'Etudes en Géophysique et Océanographie Spatiales (LEGOS)}, Toulouse, France. Working on the identification of mesoscale eddies and
eddy dynamics under the supervision of Dr. Alexis Chaigneau. This work was funded by the Research 
Internships Abroad (BEPE) program from the the São Paulo Research Foundation (FAPESP).  
Title of the project: \textit{\textbf{``The methods of identifying  mesoscale 
eddies from satellite altimetry data''}}.\\
\end{tabular}
\newline \noindent
\newline \noindent 
\newline \noindent
\begin{tabular}{L!{\VRule}R}
\textsc{2011} & {\bf Undergraduate Research - UFRN}\\
& Undergraduate research project at the Federal University of Rio Grande do Norte (UFRN), working 
on the dynamics of well-mixed estuaries.\\
\end{tabular}
\vspace{.3cm}
\section*{Publications}
\vspace{.3cm}
\bibliographystyle{plainnat}
\nobibliography{publication}
\begin{tabular}{L!{\VRule}R}
&\bibentry{villasboasandyoung}\\[5pt]
&\bibentry{merrifield2019tos}\\[5pt]
&\bibentry{villasboas2019oceanobs}\\[5pt]
\end{tabular}
\newline \noindent
\begin{tabular}{L!{\VRule}R}
&\bibentry{opencode}\\[5pt]
&\bibentry{villasboas2017california}\\[5pt]
&\bibentry{villas2015signature}\\[5pt]
&\bibentry{castelao2013objective}\\[5pt]
\end{tabular}

\vspace{.3cm}
\section*{Fellowships and Awards}
\begin{tabular}{L!{\VRule}R}
2018 & \textbf{Fellow of the Planetary Boundary Layers in Atmospheres, Oceans, and Ice on Earth and Moons Program} -- The Kavli Institute for Theoretical Physics, University of California, Santa Barbara\\[5pt]

2017 & \textbf{NASA Earth and Space Science Graduate Fellowship} -- Awarded by the National Aeronautics and Space Administration\\[5pt]

2017 & \textbf{Outstanding Mentor Award} -- Awarded by Scripps Institution of Oceanography for guidance, leadership, and unwavering commitment to helping fellow students\\[5pt]

2014 & \textbf{T.R. and Edith Folsom Endowed Graduate Fellowship Fund} -- Awarded by Scripps Institution of Oceanography\\[5pt]
\end{tabular}

\vspace{.3cm}
\section*{Service}
\begin{tabular}{L!{\VRule}R}
	2020  & \textbf{Funding agency reviewer} -- Panel reviewer for the National Aeronautics and Space Administration. \\[5pt] 
	2020  & \textbf{Session convener} -- Convener of ``\textit{Wave Breaking in Ocean-Atmosphere Exchanges}'' at the 2020 Ocean Sciences Meeting. \\[5pt] 
	2018  & \textbf{Session convener} -- Convener of ``\textit{Integrated Observations and Modeling of Surface Currents, Waves, and Winds}'' at the 2018 AGU Fall Meeting. \\[5pt] 

2016--present & \textbf{Journal Reviewer} -- Reviewer for the Journal of Physical Ocenography, the Journal of Geophysical Research, Geophysical Research Letters, and Remote sensing of Environment.\\[5pt] 

2016--present & \textbf{Outreach} -- Help lead various outreach activities at Scripps' Hydrolics Laboratory running demos in the wave tank.\\[5pt] 

2016--present & \textbf{Undergrad mentoring} -- Mentor undergraduate research projects. I currently supervise students Roger Wu (Junior, Oceanic and Atmospheric Sciences Major) and Luke Colosi (Sophomore, Oceanic and Atmospheric Sciences Major).\\[5pt] 

2016--present & \textbf{Peer Mentor} -- Mentor for first year PhD. students as part of the  peer mentor program at Scripps Institution of Oceanography, San Diego, CA.\\[5pt]

2016     & \textbf{Student Committee Member} -- Served as a member of the student committee for the observational physical oceanography faculty search at Scripps Institution of Oceanography\\[5pt]
\end{tabular}

%\section*{Workshops and Summer Schools}
%\begin{tabular}{L!{\VRule}R}
%2016 & \textbf{Software Carpentry Workshop} -- Scripps Institution of Oceanography, San Diego, CA.\\[5pt]
%2016 & \textbf{WaveWatch III Summer School} -- The Institut Français de Recherche pour l'exploitation de la Mer (IFREMER), Brest, France. Instructors: Dr. Fabrice Ardhuin and Dr. Aron Roland.\\[5pt]
%52015 & \textbf{ NASA's Earth Observations Summer School} --
%Using Satellite Observations to Advance Climate Models, Keck Institute for Space Studies, Pasadena, CA.\\[5pt]
%\end{tabular}

\section*{Teaching Experience}
\vspace{.3cm}
\begin{tabular}{L!{\VRule}R}
2019 & \textbf{Programming with Python} - School of Global Policy and Strategy, UC San Diego. \\[5pt]
2019 & \textbf{``An impractical guide to surfing surface waves''} - Guest lecture for SIO90, UC San Diego. \\[5pt]
2018 & \textbf{Software and Data Carpentry Instructor} - Certified Software and Data Carpentry Instructor. I have taught several SWC workshops for a broad range of audiences, including the Scripps Undergraduate Research Fellowship (SURF) and the UC San Diego library.  \\[5pt]
2010 & \textbf{Linear Algebra} - Teaching assistant for linear algebra -- Federal University of Rio Grande do Norte, Natal, Brazil.\\[5pt]
& \textbf{Calculus II} - Teaching assistant for calculus II -- Federal University of Rio Grande do Norte, Natal, Brazil.\\
\end{tabular}
%\vspace{.3cm}
\section*{Computational skills}
\vspace{.3cm}
\begin{tabular}{L!{\VRule}R}
\textsc{Operating Systems}& Unix-based 
operating systems, command--line, Bash, and Shell-Script. \\
\end{tabular}
\newline \noindent 
\newline \noindent
\begin{tabular}{L!{\VRule}R}
\textsc{Programming Languages}& Python, C, Fortran, and MatLab.  \\
\end{tabular}
\newline \noindent 
\newline \noindent
\begin{tabular}{L!{\VRule}R}
\textsc{Tools and Software}& LaTeX, VIM, Ansible, version control systems (Git, Mercurial), iPython notebooks, and Markdown. \\
\end{tabular}
\newline \noindent
\newline \noindent
\begin{tabular}{L!{\VRule}R}
\textsc{Numerical Modeleing}& WaveWatch III framework. \\
\end{tabular}


\vspace{.5cm}
\section*{Languages}
\vspace{.3cm}
\begin{tabular}{l l}
Portuguese: & Native language\\[3pt]
English:& Full proficiency\\[3pt] 
Spanish: & Professional working proficiency \\[3pt]
French:& Limited working proficiency \\
\end{tabular}
\vspace{.5cm}
\section*{References}
\vspace{.3cm}

\begin{minipage}[ht]{0.48\textwidth}
\begin{flushleft}
\textbf{Dr. Sarah Gille:} \\
\vspace{.1cm}
\url{sgille@ucsd.edu}\\
\vspace{.2cm}
\small{Scripps Institution of Oceanography}\\
\small{9500 Gilman Drive \#0230} \\
\small{La Jolla, CA 92093} \\
\small{+1 858--822--4425} 
\end{flushleft}
\end{minipage}
\hfill
\begin{minipage}[ht]{0.48\textwidth}
\vspace{.1cm}
\begin{flushright}
\begin{tabular}{l  l }
& \textbf{Dr. Bruce Cornuelle:} \\[5pt]
& \url{bcornuelle@ucsd.edu} \\[5pt]
&\small{Scripps Institution of Oceanography}\\
&\small{9500 Gilman Drive \#0230} \\
&\small{La Jolla, CA 92093 0230} \\
&\small{+1 858-534-4021} 
\end{tabular}
\end{flushright}
\end{minipage}




\end{document}
